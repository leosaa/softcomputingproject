\documentclass[acmsmall,nonacm,natbib=false]{acmart}
\usepackage{amsmath,amsfonts}
\usepackage{hyperref}

%%
%% \BibTeX command to typeset BibTeX logo in the docs
\AtBeginDocument{%
  \providecommand\BibTeX{{%
    Bib\TeX}}}


%%
%% The majority of ACM publications use numbered citations and
%% references, obtained by selecting the acmnumeric BibLaTeX style.
%% The acmauthoryear BibLaTeX style switches to the "author year" style.
%%
%% If you are preparing content for an event
%% sponsored by ACM SIGGRAPH, you must use the acmauthoryear style of
%% citations and references.
%%
%% Bibliography style
\RequirePackage[
  datamodel=acmdatamodel,
  style=acmnumeric,
  ]{biblatex}

%% Declare bibliography sources (one \addbibresource command per source)
\addbibresource{project-sources.bib}
\addbibresource{datasets.bib}

%%
%% end of the preamble, start of the body of the document source.
\begin{document}

%%
%% The "title" command has an optional parameter,
%% allowing the author to define a "short title" to be used in page headers.
\title{Emotional Recognition Exploration}

%%
%% The "author" command and its associated commands are used to define
%% the authors and their affiliations.
%% Of note is the shared affiliation of the first two authors, and the
%% "authornote" and "authornotemark" commands
%% used to denote shared contribution to the research.
\author{David Chamberlain}
\email{david.chamberlain@student.nmt.edu}
\affiliation{%
	\institution{New Mexico Institute of Mining and Technology}
	\city{Socorro}
	\state{New Mexico}
	\country{USA}
}

\author{Leonardo Saavedra}
\email{leonardo.saavedra@student.nmt.edu}
\affiliation{%
	\institution{New Mexico Institute of Mining and Technology}
	\city{Socorro}
	\state{New Mexico}
	\country{USA}
}

\author{Korbin Shelly}
\email{korbin.shelley@student.nmt.edu}
\affiliation{%
	\institution{New Mexico Institute of Mining and Technology}
	\city{Socorro}
	\state{New Mexico}
	\country{USA}
}

%%
%% By default, the full list of authors will be used in the page
%% headers. Often, this list is too long, and will overlap
%% other information printed in the page headers. This command allows
%% the author to define a more concise list
%% of authors' names for this purpose.
\renewcommand{\shortauthors}{Chamberlain,Saavedra,Shelly}

%%
%% The abstract is a short summary of the work to be presented in the
%% article.
\begin{abstract}
\end{abstract}

%%
%% The code below is generated by the tool at http://dl.acm.org/ccs.cfm.
%% Please copy and paste the code instead of the example below.
%%

\begin{CCSXML}
	<ccs2012>
	<concept>
	<concept_id>10002944.10011122.10002946</concept_id>
	<concept_desc>General and reference~Reference works</concept_desc>
	<concept_significance>100</concept_significance>
	</concept>
	<concept>
	<concept_id>10003120.10003121.10003125.10010597</concept_id>
	<concept_desc>Human-centered computing~Sound-based input / output</concept_desc>
	<concept_significance>500</concept_significance>
	</concept>
	<concept>
	<concept_id>10003120.10003121.10003122.10010856</concept_id>
	<concept_desc>Human-centered computing~Walkthrough evaluations</concept_desc>
	<concept_significance>100</concept_significance>
	</concept>
	<concept>
	<concept_id>10010405.10010469.10010473</concept_id>
	<concept_desc>Applied computing~Language translation</concept_desc>
	<concept_significance>300</concept_significance>
	</concept>
	</ccs2012>
\end{CCSXML}

\ccsdesc[100]{General and reference~Reference works}
\ccsdesc[500]{Human-centered computing~Sound-based input / output}
\ccsdesc[100]{Human-centered computing~Walkthrough evaluations}
\ccsdesc[300]{Applied computing~Language translation}


%%
%% Keywords. The author(s) should pick words that accurately describe
%% the work being presented. Separate the keywords with commas.
\keywords{keywords}
\maketitle
\section{abstract}
Human voice is central to almost all human activities and is the key indicator of our behaviour. Every time we talk the human voice reflecs our mood or the way we are feeling at a particular moment can often reveal our underlaying emotions in other words our emotions and sentiments. Emotional Recognition of the Voice the subject of this study. Emotional recognition has been studied for a long time based on facial expressions, but emotional recognition based on speech has only been studied in recent years, which seeks to categorize emotions according to variances in speech patterns like tone or annunciation.
\section{Introduction}

%%
%% Print the bibliography
%%
\printbibliography

%%
%% If your work has an appendix, this is the place to put it.
% \appendix
\end{document}
\endinput
